% !TeX spellcheck = pt_PT
\documentclass[a4paper]{report}
\usepackage[portuguese]{babel}
\usepackage{a4wide}

\usepackage{graphicx}
\usepackage{hyperref}
\usepackage{listings}
\usepackage{indentfirst}
\usepackage{float}
\usepackage{caption}
\usepackage{subcaption}

\setlength{\parskip}{1em}

\title{Sistemas Operativos - Trabalho Prático\\
	\large Grupo 12}
\author{Sofia Santos (A89615)
	\and Ana Filipa Pereira (A89589)
	\and Carolina Santejo (A89500)}
\date{Ano Letivo 2019/2020}

\begin{document}
	\begin{minipage}{0.9\linewidth}
        \centering
		\includegraphics[width=0.4\textwidth]{eng.jpeg}\par\vspace{1cm}
		\href{https://www.uminho.pt/PT}
		{\scshape\LARGE Universidade do Minho} \par
		\vspace{0.6cm}
		\href{https://miei.di.uminho.pt/}
		{\scshape\Large Mestrado Integrado em Engenharia Informática} \par
		\maketitle
		\begin{figure}[H]
			\includegraphics[width=0.32\linewidth]{sofia.jpg}
			\includegraphics[width=0.32\linewidth]{filipa.jpg}
			\includegraphics[width=0.32\linewidth]{santejo.jpg}
		\end{figure}
	\end{minipage}
	
	\tableofcontents
	
	\pagebreak
	
	\chapter{Introdução}
%	
	Este projeto consistiu na criação de um serviço de execução de tarefas, no qual um cliente é capaz de enviar sucessivas tarefas a um servidor, para este as executar.
	Para além de executar tarefas, o servidor permite consultar tarefas em execução ou executadas previamente, consultar o output de cada tarefa, terminar manualmente tarefas, e ainda definir um tempo máximo de execução de cada tarefa ou tempo máximo de comunicação entre pipes.
	
	Numa fase inicial, o maior desafio foi encontrar uma forma de poder encadear um número arbitrário de comandos sucessivos. Mais tarde, também tivemos alguma dificuldade em conseguir obter o estado de uma tarefa, mas acabámos por conseguir superar estes obstáculos.
	
	O cliente e o servidor comunicam através de dois pipes com nome, um envia comandos do cliente para o servidor, e o outro envia o output dos comandos do servidor para o cliente.
	
	\chapter{Funcionalidades disponíveis no servidor}
	\section{Ajuda}
	Este é o comando mais simples, que apenas envia para o cliente uma lista de todos os comandos existentes e de como os deve usar.
	
	\section{Executar}
	Comando central do servidor, permite executar uma ou mais funções encadeadas, enviando para um ficheiro \emph{log} o seu output. Neste comando usamos um ou mais \emph{execvp}, após fazer \textit{parsing} do input, para executar cada função fornecida. Estas funções comunicam entre si através de pipes anónimos. Como não sabemos de quantos pipes uma tarefa irá precisar antes de terminar o \textit{parsing} total do input, por uma questão de eficiência, decidimos dar a cada tarefa 32 pipes anónimos. Podíamos também ter usado alocação dinâmica de memória, para por um lado evitar usar memória desnecessariamente e por outro permitir a execução de comandos com mais de 32 pipes, mas acreditamos que, para além de que o número atual de pipes que temos não ter um grande peso em termos de memória, 32 já é um número mais que suficiente de pipes para uma tarefa.

	\section{Histórico}
	O servidor contém duas listas, uma que contém todas as tarefas executadas até ao momento, e outra que contém o estado da execução de cada tarefa. Este estado pode ser: em execução; concluída; terminada manualmente; terminada por tempo de execução; terminada por inatividade. O comando \emph{historico} envia para o cliente uma lista contendo a informação destas duas listas.
	
	\section{Listar}
	Este comando é uma versão mais simples do comando \emph{historico}, apenas envia para o cliente as tarefas do histórico que ainda estejam em execução.
	
	\section{Output}
	Todas as tarefas que terminam normalmente escrevem o seu output no ficheiro \emph{log}, e é guardado no ficheiro \emph{log.idx}, associada ao número da tarefa, a posição do byte final do output no \emph{log}. O comando \emph{output} lê o ficheiro \emph{log} e envia para o cliente o output da tarefa especificada, começando a ler a partir do fim do output da tarefa anterior (posição 0 caso seja a primeira tarefa) até à posição indicada no ficheiro \emph{log.idx}. Com este sistema apenas temos que armazenar o número da tarefa e uma posição num ficheiro auxiliar.
	
	\section{Terminar}
	Como uma tarefa pode gerar vários processos filhos, é necessário guardar os pids de todos eles, para os podermos terminar. Para isso usamos uma lista, que o comando \emph{terminar} percorre, e para cada pid de um processo filho de uma dada tarefa usa a função \emph{kill} para o terminar, usando o sinal \emph{SIGTERM}.
	
	\section{Tempo de execução}
	Este comando permite-nos definir um tempo máximo durante o qual cada tarefa pode correr antes de ser terminada à força. O programa guarda numa lista o tempo de execução de cada tarefa, e a cada segundo verifica, caso a tarefa ainda esteja em execução, se esse tempo é superior ao tempo máximo de execução. Em caso afirmativo, essa tarefa é terminado pelo mesmo método do comando \emph{terminar} e o seu estado atualizado. Caso não seja, o tempo de execução dessa tarefa é incrementado em um segundo. Esta sistema funciona usando a função \emph{alarm}, que é chamada uma vez por segundo, e faz essa verificação.
	
	\section{Tempo de inatividade}
	Ao dar a flag \emph{O\_ASYNC} aos pipes de comunicação das tarefas, estes enviam um sinal (\emph{SIGIO}) quando passa informação por eles. Desta forma, somos capazes de saber quando é que houve comunicação num pipe anónimo. Este comando usa o mesmo método de medição de tempo que o comando \emph{tempo-execucao}, mas este apenas interrompe uma tarefa se não tiver havido comunicação nos pipes dessa tarefa. O comando sabe isso através de uma variável que passa a ser zero sempre que há comunicação num dos pipes da tarefa, e que é incrementada a cada segundo. Desta forma, se chegar ao valor do tempo de inatividade, é porque não houve atividade durante aquele tempo. Como é necessário saber a que tarefa pertence o pipe que enviou o sinal, os sinais são enviados para um processo filho "monitor". Cada tarefa tem um monitor, que, se receber o sinal SIGIO, envia outro sinal para o processo argusd, que consegue detetar qual o monitor que enviou o sinal e terminar a respetiva tarefa. Estes monitores são terminados sempre que uma tarefa termina, seja naturalmente ou não.
		
	\chapter{Testes}
	
	Apresentamos agora alguns dos testes manuais que fizemos para testar o funcionamento do nosso programa. Para além destes fizemos mais testes e testes mais complexos, mas estes cobrem todas as funcionalidades principais do programa. Nas imagens abaixo, se não houver indicação em contrário, os comandos foram executados sequencialmente, um imediatamente após o outro.
	
	\begin{figure}[H]
		\centering
		\begin{subfigure}{.3\textwidth}
			\centering
			\includegraphics[width=.9\linewidth]{teste1.png}
			\caption{Teste 'executar', 'historico' e 'output'.}
			\label{fig:test1}
		\end{subfigure}
		\begin{subfigure}{.3\textwidth}
			\centering
			\includegraphics[width=.9\linewidth]{teste2.png}
			\caption{Teste 'listar' e 'terminar'.}
			\label{fig:test2}
		\end{subfigure}
		\begin{subfigure}{.3\textwidth}
			\centering
			\includegraphics[width=.9\linewidth]{teste3.png}
			\caption{Teste 'tempo-execucao'.}
			\label{fig:test3}
		\end{subfigure}
	\end{figure}
		
	\pagebreak
	Para testar o comando \emph{tempo-inactividade} fizemos um pequeno programa em \emph{Python} que recebe input pela linha de comandos ou pelo stdin e o envia como output passados X segundos (10 por defeito). Desta forma, ao encadear 3 invocações deste programa, por exemplo, temos uma tarefa que demora 30 segundos a executar, mas cujos pipes apenas estão inativos por períodos de 10 segundos. No exemplo abaixo, a mesma tarefa apenas é terminada quando o tempo de inatividade é inferior a 10 segundos.
	\begin{figure}[H]
		\centering
		\includegraphics[width=.9\linewidth]{teste4.png}
		\caption{Teste 'tempo-inactividade'.}
		\label{fig:test4}
	\end{figure}
		
	\chapter{Decisões tomadas}
	
	O nosso ficheiro de log, onde são armazenados os outputs das tarefas, é apagado sempre que o servidor é reiniciado. Decidimos fazer desta forma, mas seria muito simples fazer de outra forma, apenas fizemos assim por escolha e porque o enunciado não referia que método tínhamos que usar. Desta forma, cada execução do servidor irá produzir um novo ficheiro de log.
	
	Decidimos que o nosso servidor suporta a execução de até 2048 tarefas. Podíamos ter usado alocação dinâmica de memória para suportar a execução de mais tarefas, mas sentimos que não seria necessário, já que, tendo em conta a simplicidade do programa, seria quase inconcebível executar mais que algumas centenas de tarefas, o histórico seria demasiado grande. Num projeto de mais larga escala, com um histórico navegável por páginas, por exemplo, isso já seria algo que implementaríamos.
	
	Para além disso, cada comando enviado ao servidor pode conter até 4098 caracteres. Contudo, este valor pode ser facilmente alterado no ficheiro \emph{argus.h}.

	\chapter{Conclusão}
	
	Como é possível constatar, o nosso trabalho possui todas as funcionalidades pedidas a funcionar perfeitamente, para além da funcionalidade adicional. Desta forma, acreditamos que o nosso trabalho está bem conseguido. Tal como referimos no capítulo anterior, podíamos ter feito algumas coisas de forma diferente, mas da forma que fizemos funciona tudo bem.


\end{document}